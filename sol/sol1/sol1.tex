\documentclass[12pt, notitlepage]{article}
\usepackage{amsmath}
\usepackage{amssymb}
\usepackage{amsthm}
\usepackage{listings}
\usepackage{color}

\definecolor{dkgreen}{rgb}{0,0.6,0}
\definecolor{gray}{rgb}{0.5,0.5,0.5}
\definecolor{mauve}{rgb}{0.58,0,0.82}

\lstset{
	frame=single,
	language=Java,
	belowskip=3mm,
	showstringspaces=false,
	columns=flexible,
	captionpos=b,
	basicstyle={\small\ttfamily},
	numbers=left,
	numbersep=5pt,
	%numbers=none,
	numberstyle=\tiny\color{gray},
	keywordstyle=\color{blue},
	commentstyle=\color{dkgreen},
	stringstyle=\color{mauve},
	breaklines=true,
	breakatwhitespace=true,
	tabsize=3
}


\providecommand{\abs}[1]{\lvert#1\rvert}
\providecommand{\norm}[1]{\lVert#1\rVert}

\newtheorem{thm}{Theorem}
\newtheorem{lemma}[thm]{Lemma}
\newtheorem{fact}[thm]{Fact}
\newtheorem{cor}[thm]{Corollary}
\newtheorem{eg}{Example}
\newtheorem{ex}{Exercise}
\newtheorem{defi}{Definition}
\newtheorem{hw}{Homework}
\newenvironment{sol}
  {\par\vspace{3mm}\noindent{\it Solution}.}

\newcommand{\fib}{\mbox{fib}}
\newcommand{\ov}{\overline}
\newcommand{\cb}{{\cal B}}
\newcommand{\cc}{{\cal C}}
\newcommand{\cd}{{\cal D}}
\newcommand{\ce}{{\cal E}}
\newcommand{\cf}{{\cal F}}
\newcommand{\ch}{{\cal H}}
\newcommand{\cl}{{\cal L}}
\newcommand{\cm}{{\cal M}}
\newcommand{\cp}{{\cal P}}
\newcommand{\cz}{{\cal Z}}
\newcommand{\eps}{\varepsilon}
\newcommand{\ra}{\rightarrow}
\newcommand{\la}{\leftarrow}
\newcommand{\Ra}{\Rightarrow}
\newcommand{\dist}{\mbox{\rm dist}}
\newcommand{\bn}{{\mathbf N}}
\newcommand{\bz}{{\mathbf Z}}

\setlength{\parindent}{0pt}
%\setlength{\parskip}{2ex}
\newenvironment{proofof}[1]{\bigskip\noindent{\itshape #1. }}{\hfill$\Box$\medskip}

\usepackage{enumerate,fullpage, proof}
\newcommand{\Infer}[2]{\infer{#2}{#1}}

\title{Homework 1}
\author{Team: nogg\footnote{E-mail: \texttt{kimi.ysma@gmail.com}}\footnote{Team member: Ma Yesheng, Zhao Min, Hu Hu, Zou Yikai, Fan Minghua}}

\begin{document}

{\bf\small CS214: Algorithms and Complexity}\hfill{\bf\small 2016 Fall}
{\let\newpage\relax\maketitle}





\begin{ex}\rm
\end{ex}

\begin{sol}
	\ \\ \\
	Pseudocode:
	
	\begin{lstlisting}
	function binomialCoefficient:
		if k < 0 or k > n: return 0;
		if k == 0 or k == n: return 1;
		return C(n-1, k-1) + C(n-1, k);
	\end{lstlisting}
	\ \\
	Python code:
	\begin{lstlisting}[language=Python]
	def biCoef(n, k):
		if k < 0 or k > n:
        	return 0
    	if k == 0 or k == n:
			return 1
    	return biCoef(n-1, k) + biCoef(n-1, k-1)
	\end{lstlisting}
	\ \\
	The running time is $O(C(n,\ k))$.Actually, it takes C(n, k)-1 times of addition.\\
	It may be quite efficient when k comes very close to zero or n, while it's not when k is close to $[n/2]$. \\
	
\end{sol}

\begin{ex}\rm
\end{ex}

\begin{sol}
	\ \\ \\
	Pseudocode:
	
	\begin{lstlisting}
	function binomialCoefficient:
		if k < 0 or k > n: return 0;
		
		for i from 0 to n:
			for j from 0 to min(i, k):
				if j == 0 or j == i:
					C(i,j) = 1;
				else:
					C(i, j) = C(i - 1, j - 1) + C(i - 1, j);
					
		return C(n, k);
	\end{lstlisting}
	\ \\
	Python code:
	\begin{lstlisting}[language=Python]
	def biCoef(n, k):
		if k < 0 or k > n:
			return 0
		
		C = [[0 for i in range(k + 1)] for i in range(n + 1)]
		for i in range(n + 1):
			for j in range(min(i, k) + 1):
				if j == 0 or j == i:
					C[i][j] = 1
				else:
					C[i][j] = C[i - 1][j - 1] + C[i - 1][j]
	
		return C[n][k]
	\end{lstlisting}
	\ \\
	The running time is $O(n * k)$.\\
	It is efficient but it can be better. We can use below optimization to reduce space use.\\ \\
	Optimal Python code:
	\begin{lstlisting}[language=Python]
	def biCoef(n , k):
		if k < 0 or k > n:
			return 0
		
		C = [0 for i in range(k + 1)]
		C[0] = 1
		
		for i in range(1, n + 1):
			j = min(i ,k)
			while (j > 0):
				C[j] = C[j] + C[j - 1]
				j -= 1
	
		return C[k]
	\end{lstlisting}
	
\end{sol}

\begin{ex}\em
\end{ex}

\begin{sol}
	\ \\ \\
	Running time is $O(n * k + \text{numberOfBit} ^ 2)$. \\
	It's not so efficient. Because when $n$ and $k$ are very large, it cost lot of time and space to calculate the result.\\
	
\begin{ex}\rm
\end{ex}
	
No such exercise.

\begin{ex}\rm
\end{ex}
\emph{Solution:}\\
The code that uses property of Sierpinski triangle is as follows:
\begin{lstlisting}[language=python]
def coeffMod2(n, k):
	if k < 0 or k > n:
		return 0
	if k == 0 and n == 0:
		return 1
	nBitLen = n.bit_length()
	kBitLen = k.bit_length()
	if n-k >= 1 and n >= 2**max(nBitLen, kBitLen) and k < 2**max(nBitLen, kBitLen):
		return 0
	elif k < 2**(nBitLen-1):
		return coeffMod2(n-2**(nBitLen-1), k)
	else:
		return coeffMod2(n-2**(nBitLen-1), k-2**(nBitLen-1))
\end{lstlisting}


\begin{ex}\rm
\end{ex}

\emph{Solution:}\\
From the fibonacci's definition: $F_0=0, F_1=1, F_n = F_{n-1} + F_{n-2}(n\geq2)$.\\
We can infer that if $F_i=F_j,F_{i+1}=F_{j+1}$, then $F_{i+2}=F_{j+2}$.\\
And naturally, we have the conclusion that if $F_i\equiv F_j(mod~k),F_{i+1}\equiv F_{j+1}(mod~k)$,then $F_{i+2}\equiv F_{j+2}(mod~k)$.\\

For $F_n\prime :=(F_n~mod~k)$, we have known that there must be some $i~and~j(i>j)~between~0~and~k^{2}$ for which $F_i\prime=F_j\prime, F_{i+1}\prime = F_{j+1}\prime$.\\
So we can infer that $F_{i-1}\prime = F_{j-1}\prime$.\\
And going on...$F_{i-2}\prime = F_{j-2}\prime$.\\
...\\
$F_{i-j+1}\prime = F_{1}\prime$.\\
$F_{i-j}\prime = F_{0}\prime$.\\
So we have the cycle from $0 to i-j-1$.\\
And make $i-j := j$,then we have the conclusion:\\
For some j we have $F_0\prime=F_j\prime,F_1\prime=F_{j+1}\prime$ and thus $F_{n}\prime=F_{n~mod~j}\prime$\\

\begin{ex}\rm
\end{ex}
\emph{Solution:}\\
Are the lower bounds of the algorithms for computing multiplication and division the  same?
\end{sol}

\end{document}
