\documentclass[12pt, notitlepage]{article}
\usepackage{amsmath}
\usepackage{amssymb}
\usepackage{graphicx}
\usepackage{amsthm}
\usepackage{listings}
\usepackage{color}
\usepackage{float}

\definecolor{dkgreen}{rgb}{0,0.6,0}
\definecolor{gray}{rgb}{0.5,0.5,0.5}
\definecolor{mauve}{rgb}{0.58,0,0.82}

\lstset{
	frame=single,
	language=Java,
	belowskip=3mm,
	showstringspaces=false,
	columns=flexible,
	captionpos=b,
	basicstyle={\small\ttfamily},
	numbers=left,
	numbersep=5pt,
	%numbers=none,
	numberstyle=\tiny\color{gray},
	keywordstyle=\color{blue},
	commentstyle=\color{dkgreen},
	stringstyle=\color{mauve},
	breaklines=true,
	breakatwhitespace=true,
	tabsize=3
}


\providecommand{\abs}[1]{\lvert#1\rvert}
\providecommand{\norm}[1]{\lVert#1\rVert}

\newtheorem{thm}{Theorem}
\newtheorem{lemma}[thm]{Lemma}
\newtheorem{fact}[thm]{Fact}
\newtheorem{cor}[thm]{Corollary}
\newtheorem{eg}{Example}
\newtheorem{ex}{Exercise}
\newtheorem{defi}{Definition}
\newtheorem{hw}{Homework}
\newenvironment{sol}
  {\par\vspace{3mm}\noindent{\it Solution}.}{\qed}

\newcommand{\fib}{\mbox{fib}}
\newcommand{\ov}{\overline}
\newcommand{\cb}{{\cal B}}
\newcommand{\cc}{{\cal C}}
\newcommand{\cd}{{\cal D}}
\newcommand{\ce}{{\cal E}}
\newcommand{\cf}{{\cal F}}
\newcommand{\ch}{{\cal H}}
\newcommand{\cl}{{\cal L}}
\newcommand{\cm}{{\cal M}}
\newcommand{\cp}{{\cal P}}
\newcommand{\cz}{{\cal Z}}
\newcommand{\eps}{\varepsilon}
\newcommand{\ra}{\rightarrow}
\newcommand{\la}{\leftarrow}
\newcommand{\Ra}{\Rightarrow}
\newcommand{\dist}{\mbox{\rm dist}}
\newcommand{\bn}{{\mathbf N}}
\newcommand{\bz}{{\mathbf Z}}

\setlength{\parindent}{0pt}
%\setlength{\parskip}{2ex}
\newenvironment{proofof}[1]{\bigskip\noindent{\itshape #1. }}{\hfill$\Box$\medskip}

\usepackage{enumerate,fullpage, proof}
\newcommand{\Infer}[2]{\infer{#2}{#1}}

\title{Homework 3}
\author{Team: nogg\footnote{E-mail: \texttt{kimi.ysma@gmail.com}}\footnote{Team member: Ma Yesheng, Zhao Ming, Hu Hu, Zou Yikai, Fan Minghua}}

\begin{document}

{\bf\small CS214: Algorithms and Complexity}\hfill{\bf\small 2016 Fall}
{\let\newpage\relax\maketitle}


\textbf{Exercise 1}
\begin{enumerate}
\item

\begin{defi}[Cut Lemma]
\vspace{-0.85cm}
Suppose edge set $X$ is good, pick any vertex set $S\subseteq V$ s.t. there is no edge goes from $S$ to $V\backslash S$. Let $e\in E$ be the edge going from $S$ to  $V\backslash X$ with the cheapest weight, then $X\cup \{e\}$ is also good.
\end{defi}
\begin{proof}
	\mbox{ }
\begin{enumerate}[(1)]
	\item If the cheapest edge $e$ happens to be in the tree $T$, then the case is trivial.
	\item If the cheapest edge $e$ is not in the tree $T$, since $T$ is already a tree, adding any edge to it will result in a circle and there must exist another edge $e'$ which also goes from $S$ to  $V\backslash X$. If we remove this edge $e'$, we will get another graph $T' = T\cup \{e\} -\{e'\}$. Next, we are going to prove that it is also a minimum spanning tree.
	\begin{enumerate}[(a)]
		\item First, we prove that $T'$ is a tree. Since $T$ is a tree, adding a edge to it will form a circle. Then we remove the  edge $e'$ from $T\cup\{e\}$ where $e'$ is part of a circle and removing it will not disconnect the graph, hence $T' = T\cup\{e\} - e'$ is also connected. On the other hand, in the connected graph $T'$, $|E|-|V| = 1$, therefore $T'$ is a tree.
		\item Next, we prove that $T'$ is a minimum spanning tree. Since substitute $e'$ for $e$ will not affect spanning property of minimum spanning tree, all we need to prove is it takes minimum weight. From the equation $weight(T') = weight(T)-w(e)+w(e')$, since $e'$ is chosen to be the edge with minimum weight, thus $weight(T') < weight(T)$. Therefore $T'$ is a minimum spanning tree.
		
	\end{enumerate}
\end{enumerate}
Combine (1) and (2), cut lemma is proved.
\end{proof}


\end{enumerate}


\end{document}
