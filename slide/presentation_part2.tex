\documentclass{beamer}

\mode<presentation> {

% The Beamer class comes with a number of default slide themes
% which change the colors and layouts of slides. Below this is a list
% of all the themes, uncomment each in turn to see what they look like.


%\usetheme{default}
%\usetheme{AnnArbor}
%\usetheme{Antibes}
%\usetheme{Bergen}
%\usetheme{Berkeley}
%\usetheme{Berlin}
%\usetheme{Boadilla}
%\usetheme{CambridgeUS}
%\usetheme{Copenhagen}
%\usetheme{Darmstadt}
%\usetheme{Dresden}
%\usetheme{Frankfurt}
%\usetheme{Goettingen}
%\usetheme{Hannover}
%\usetheme{Ilmenau}
%\usetheme{JuanLesPins}
%\usetheme{Luebeck}
%\usetheme{Madrid}
%\usetheme{Malmoe}
%\usetheme{Marburg}
%\usetheme{Montpellier}
%\usetheme{PaloAlto}
\usetheme{Pittsburgh}
%\usetheme{Rochester}
%\usetheme{Singapore}
%\usetheme{Szeged}
%\usetheme{Warsaw}

% As well as themes, the Beamer class has a number of color themes
% for any slide theme. Uncomment each of these in turn to see how it
% changes the colors of your current slide theme.

%\usecolortheme{albatross}
%\usecolortheme{beaver}
%\usecolortheme{beetle}
%\usecolortheme{crane}
%\usecolortheme{dolphin}
%\usecolortheme{dove}
%\usecolortheme{fly}
%\usecolortheme{lily}
%\usecolortheme{orchid}
%\usecolortheme{rose}
%\usecolortheme{seagull}
%\usecolortheme{seahorse}
%\usecolortheme{whale}
%\usecolortheme{wolverine}

%\setbeamertemplate{footline} % To remove the footer line in all slides uncomment this line
%\setbeamertemplate{footline}[page number] % To replace the footer line in all slides with a simple slide count uncomment this line

%\setbeamertemplate{navigation symbols}{} % To remove the navigation symbols from the bottom of all slides uncomment this line
}

\usepackage{graphicx}
\usepackage{booktabs}
\usepackage{amssymb}
\usepackage{color}

\title[Short title]{Non-zero-sum Game and Nash Equilibarium}

\author{Team nogg}
\institute[SJTU]
{

}
\date{\today}

\begin{document}

\begin{frame}
\titlepage
\end{frame}

\begin{frame}
\frametitle{Overview}
\tableofcontents
\end{frame}


\section{Useful Concepts}
\begin{frame}
\frametitle{What is a Game?}
A normal-form game:\\
\begin{itemize}
\item The {\color{red} number} of players n: $[n] = \{1,2,...,n\}$
\item a finite {\color{red} set of pure strategies} $S_p$ for each player p
\item a {\color{red} utility function} $u_p$ for each player p\\
        \qquad $u_p : S_1 \times S_2 \times ... \times S_n  \rightarrow \mathbb{R}$\\
        In General words, you can get the pay-off of player p using the function $u_p$ when choosing a concrete strategy combination.
\end{itemize}
\end{frame}

\begin{frame}
\frametitle{What is a Game?}
\begin{itemize}
\item A normal-form game can be summarized as a tuple:\\
        \qquad $<n, (S_p)_{p\in [n]}, (u_p)_{p\in [n]}>$
\item Complete information: every player know the whole tuple.
\end{itemize}

\end{frame}

\begin{frame}
\frametitle{Pure Strategy and Mixed Strategy}
Pure Strategy:\\
\begin{itemize}
\item Under complete information, if every player will only choose one strategy, then it is a pure strategy.
\item Pure strategy's pay-off can be obtained by the utility function $u_p$.
\item It is a special case of mixed strategy.
\end{itemize}
\end{frame}

\begin{frame}
\frametitle{Pure Strategy Example}
Prisoner's Dilemma:
\begin{itemize}
\item
\begin{tabular}{|c|c|c|}
\hline
\hline
    & B cooperates & B defects\\
\hline
A cooperates & (-1,-1) & (-9,0)\\
\hline
A defects & (0,-9) & (-6,-6)\\
\hline
\hline
\end{tabular}
\item
(x,y) means A's pay-off is x and B's pay-off is y.\\
\item "A defects and B defects" is a pure strategy.\\
\item Actually, this pure strategy is a Nash Equilibrium.
\end{itemize}
\end{frame}

\begin{frame}
\frametitle{Pure Strategy and Mixed Strategy}
Mixed Strategy:\\
\begin{itemize}
\item Under complete information, if player will choose strategies based on some probability, then it is a mixed strategy.
\item Mixed strategy's pay-off can only be predicted.
\end{itemize}
\end{frame}

\begin{frame}
\frametitle{Mixed Strategy Example}
Guessing Game:\\
There are two persons. Each of them has a coin, they are going to show an edge together.\\
\begin{itemize}
\item
\begin{tabular}{|c|c|c|}
\hline
\hline
    & B head & B tail\\
\hline
A head & (1,-1) & (-1,1)\\
\hline
A tail & (-1,1) & (1,-1)\\
\hline
\hline
\end{tabular}
\item
(x,y) means A's pay-off is x and B's pay-off is y.\\
\item After thinking about the game, "A and B both choose to show head of the coin at 50\% probability and show tail at 50\% probability", it is a mixed strategy.
\item Actually, this mixed strategy is a Nash Equilibrium.
\end{itemize}
\end{frame}

\begin{frame}
\frametitle{What is Nash Equilibrium?}

In words, x is a Nash equilibrium iff no player can strictly increase his or her payoff by switching to
a different mixed strategy, if the other players don��t change their strategies.
\end{frame}



\end{document}
